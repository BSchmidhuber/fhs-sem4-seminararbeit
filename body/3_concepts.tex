\section{Concepts and Techniques}


\subsection{Fairlets}

At first glance, the fair version of a clustering problem appears harder than its vanilla counterpart. \autocite[5]{Chierichetti2018} But there is a simple solution to this. \textcite[5]{Chierichetti2018} prove a reduction from the former to the latter by first clustering the original points into small clusters preserving the balance, and then applying vanilla clustering on these smaller clusters instead of on the original points.

So they introduced the term "fairlet", to denote fair clusters with a minimal set "while approximately preserving the clustering objective". \autocite[]{Chierichetti2018}


\subsection{Stochastic Block Models}
% TODO WIKIPEDIA
The stochastic block model is a generative model for random graphs. This model tends to produce graphs containing communities, subsets characterized by being connected with one another with particular edge densities. For example, edges may be more common within communities than between communities. The stochastic block model is important in statistics, machine learning, and network science, where it serves as a useful benchmark for the task of recovering community structure in graph data.
% TODO WIKIPEDIA END

A stochastic model is proposed for social networks in which the actors in a network are partitioned into subgroups called blocks. The model provides a stochastic generalization of the blockmodel. \autocite[]{Holland1983}


Other References: \textcite[]{Lei2013}

\subsection{Normalized Cut}
Normalized Cut is an algorithm for image segmentation, which is mainly used in clustering algorithms, such as the spectral clustering.
As \textcite[]{Eriksson2011} said, Indisputably Normalized Cuts is one of the most popular segmentation algorithms in computer vision.

Other References: 
\textcite[]{Xu2010},
\textcite[]{Yu2004},
\textcite[]{Shi2000}