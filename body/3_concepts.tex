\section{Concepts and Techniques}


\subsection{Fairlets}
\label{fairlets}

The first concept presented in this paper is the concept of fairlets. The term "fairlet" was introduced by  \textcite[]{Chierichetti2018}, to denote "minimal sets that satisfy fair representation
while approximately preserving the clustering objective." \autocite[1]{Chierichetti2018}

\subsubsection{Balance}

To understand the concept of fairlets, it is first necessary to understand the concept of balance. Assuming we have a data set of red and blue data points, balance is the quotient of the smaller by the larger value, which must be between 0 and 1. If we have a completely balanced distribution of data points, e.g. 5 red and 5 blue points, the balance is 1, which means that the data set is perfectly balanced. If it is relatively unbalanced, e.g. 1 red data point and 10 blue, the balance is $\frac{1}{10}$ = 0.1. If we don't have any points at all from one color, the balance is 0, so fully unbalanced. This balance can also be combined in disjoint subsets by using the minimum balance of each case, which provides a basis for the concept of fairlets. \autocite[4]{Chierichetti2018}

\begin{quote}
"Balance encapsulates a specific notion of fairness, where a clustering with a monochromatic cluster (i.e., fully unbalanced) is considered unfair."

\autocite[4]{Chierichetti2018}
\end{quote}

\subsubsection{Fairlet decomposition}

Once the balance has been determined, a clustering can be constructed iteratively from the original data set. For that, single data points are removed as long as the balance between the remaining data points can be kept the same. To bring fairness into play, this process to get a minimum data set with the same balance is called fairlet decomposition, because the so-called ($t, k$)-fairlets are decomposed until you get $k$ desired clusters that have reached a corresponding balance of $t$. \autocite[4]{Chierichetti2018}

Finally, \textcite[4]{Chierichetti2018} state that "balance encapsulates a specific notion of fairness, where a clustering with a monochromatic cluster (i.e., fully unbalanced) is considered unfair. So we call the clustering [...] a $(b, r)$\textit{-fairlet decomposition} [...] and call each cluster [...] \textit{a fairlet}." \autocite[4]{Chierichetti2018}


\subsection{Stochastic Block Models}

Stochastic Block Models were introduced and defined by \textcite[]{Holland1983}:

\begin{quote}
"A stochastic model is proposed for social networks in which the actors in a network are partitioned into subgroups called blocks. The model provides a stochastic generalization of the blockmodel."

\autocite[1]{Holland1983}
\end{quote}

\textcite[1]{Karrer2010StochasticNetworks} describe it as a "generative model for blocks, groups, or communities in networks." Classified as "random graph models" they already exist for a long time in the area of computer science.
The simplest form of a stochastic block model is that "each of $n$ vertices is assigned to one of $K$ blocks, groups, or communities, and undirected edges are placed independently between vertex pairs with probabilities that are a function only of the group memberships of the vertices." \autocite[1]{Karrer2010StochasticNetworks}

In the case of clustering, \textcite[]{Lei2013} describe the consistency of spectral clustering in stochastic block models. They "show that, under mild conditions, spectral clustering applied to the adjacency matrix of the network can consistently recover hidden communities even when the order of the maximum expected degree is as small as log $n$, with $n$ the number of nodes." \autocite[1]{Lei2013}
