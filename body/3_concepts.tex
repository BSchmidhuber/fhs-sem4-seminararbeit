\section{Concepts and Techniques}


\subsection{Fairlets}
\label{fairlets}

The first concept presented in this paper is the concept of fairlets. The term "fairlet" was introduced by  \textcite[]{Chierichetti2018}, to denote "minimal sets that satisfy fair representation
while approximately preserving the clustering objective." \autocite[1]{Chierichetti2018} 

\textcite[]{Chierichetti2018} say, that fair clustering problems appear to be harder as the normal clustering problems without fairness constraints. They "prove a reduction from the former to the latter" and did this "by first clustering the original points into small clusters preserving the balance, and then applying vanilla clustering on these smaller clusters instead of on the original points." \autocite[5]{Chierichetti2018}


\subsection{Stochastic Block Models}

Stochastic Block Models were introduced and defined by \textcite[]{Holland1983}:

\begin{quote}
"A stochastic model is proposed for social networks in which the actors in a network are partitioned into subgroups called blocks. The model provides a stochastic generalization of the blockmodel."

\autocite[1]{Holland1983}
\end{quote}

\textcite[1]{Karrer2010StochasticNetworks} describe it as a "generative model for blocks, groups, or communities in networks." Classified as "random graph models" they already exist for a long time in the area of computer science.
The simplest form of a stochastic blockmodel is that "each of $n$ vertices is assigned to one of $K$ blocks, groups, or communities, and undirected edges are placed independently between vertex pairs with probabilities that are a function only of the group memberships of the vertices." \autocite[1]{Karrer2010StochasticNetworks}

In the case of clustering, \textcite[]{Lei2013} describe the consistency of spectral clustering in stochastic block models. They "show that, under mild conditions, spectral clustering applied to the adjacency matrix of the network can consistently recover hidden communities even when the order of the maximum expected degree is as small as log $n$, with $n$ the number of nodes." \autocite[1]{Lei2013}


\subsection{Normalized Cut}

As a solution for segmenting graph data, \textcite[]{Shi2000} proposed normalized cuts, an "approach for solving the perceptual grouping problem in vision." \autocite[1]{Shi2000} This solution became famous in the area of unsupervised machine learning, as \textcite[]{Eriksson2011} said, "Indisputably Normalized Cuts is one of the most popular segmentation algorithms in computer vision. It has been applied to a wide range of segmentation tasks with great success." \autocite[1]{Eriksson2011}
