\section{Definitions}


\subsection{Clustering}

When it comes to data analysis in the area of unsupervised machine learning, \textcite[1]{VonLuxburg2007} says that "Clustering is one of the most widely used techniques for exploratory data analysis [...]". For this purpose, data is analyzed so that groups with similar behavior can be identified and summarized.

\textcite[1]{Bera2019} sums it up with the addition, that "Clustering is a fundamental unsupervised learning problem where one wants to partition a given data-set. In machine learning, clustering is often used for feature generation and enhancement as well. It is thus important to consider the bias and unfairness issues when inspecting the quality of clusters."

"Considerable research in machine learning and pattern recognition" has the focus on "the task of finding good clusters". \autocite[1]{Ng2001}

\textcite[1]{Nascimento2011} add, that "Clustering is an unsupervised technique concerned with the grouping of related objects without taking their class or label into account."

There are various \nameref{clustering-algorithms} that we will have a look at in \autoref{clustering-algorithms}.


\subsection{Fairness}

To define fairness in the area of machine learning \textcite[1]{Kleindessner2019} said, that "[...] a clustering is fair if every demographic group is approximately proportionally represented in each cluster."

\textcite[1]{Feldman2014CertifyingImpact} raises the question of biased algorithms and claims that "unintentional bias" is called "disparate impact, which occurs when a selection process has widely different outcomes for different groups, even as it appears to be neutral." There is a problem of "determining disparate impact", especially "when computers are involved".

